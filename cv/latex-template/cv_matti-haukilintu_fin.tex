%%%%%%%%%%%%%%%%%%%%%%%%%%%%%%%%%%%%%%%%%
% Twenty Seconds Resume/CV
% LaTeX Template
% Version 1.1 (8/1/17)
%
% This template has been downloaded from:
% http://www.LaTeXTemplates.com
%
% Original author:
% Carmine Spagnuolo (cspagnuolo@unisa.it) with major modifications by
% Vel (vel@LaTeXTemplates.com)
%
% License:
% The MIT License (see included LICENSE file)
%
%%%%%%%%%%%%%%%%%%%%%%%%%%%%%%%%%%%%%%%%%

%----------------------------------------------------------------------------------------
%	PACKAGES AND OTHER DOCUMENT CONFIGURATIONS
%----------------------------------------------------------------------------------------

\documentclass[letterpaper]{twentysecondcv}

%----------------------------------------------------------------------------------------
%	 TRANSLATED
%----------------------------------------------------------------------------------------

\RequirePackage[finnish]{optional}

%----------------------------------------------------------------------------------------
%	 PERSONAL INFORMATION
%----------------------------------------------------------------------------------------

% If you don't need one or more of the below, just remove the content leaving the command, e.g. \cvnumberphone{}

\profilepic{../../assets/images/profile_bw.jpg} % Profile picture

\cvname{Matti Haukilintu} % Your name
\cvjobtitle{Sovellusasiantuntija} % Job title/career

\cvdate{28.4.1990} % Date of birth
\cvaddress{Jyväskylä, Suomi Finland} % Short address/location, use \newline if more than 1 line is required
\cvnumberphone{+358 504961581} % Phone number
\cvsite{http://matson.haukilintu.fi} % Personal website
\cvgit{https://github.com/Amatson} % Git page
\cvmail{matti@haukilintu.fi} % Email address

%----------------------------------------------------------------------------------------

\begin{document}

%----------------------------------------------------------------------------------------
%	 ABOUT ME
%----------------------------------------------------------------------------------------

\aboutme{
Matti on aktiivinen ja uudesta kiinnostuva sovelluskehittäjä,
jolla on laaja-alainen kokemus IT-alalta.
} % To have no About Me section, just remove all the text and leave \aboutme{}

%----------------------------------------------------------------------------------------
%	 SKILLS
%----------------------------------------------------------------------------------------

% Skill bar section, each skill must have a value between 0 an 6 (float)
\skills{{Python/5},{C++/4},{C/3},{C\#/4},{Java/4.7},{SQL/5.9},{Web-teknologiat/5.2}}

%------------------------------------------------

% Skill text section, each skill must have a value between 0 an 6
\skillstext{{Sosiaalinen/6},{Tiimityö/6},{Sopeutuva/5},{Huumori/6},{Luovuus/5}}

%----------------------------------------------------------------------------------------

\quoteblock{Thrust the process!}

%----------------------------------------------------------------------------------------

\makeprofile % Print the sidebar

%----------------------------------------------------------------------------------------
%	 EXPERIENCE
%----------------------------------------------------------------------------------------

\section{Työkokemus}

\begin{twenty} % Environment for a list with descriptions
	\twentyitem{2019-}{Sovellusasiantuntija}{Samlink Oy}
    {Sovelluskehitysprojekteja pankkialalla, \\
		sekä DNA Viihdepalvelun ja web-käyttöliittymän kehitystä.\\
		\faIcon{carrot} \texttt{Java}, \texttt{Android}, \texttt{Node.js}, \texttt{Angular}, \texttt{Typescript} \\
	Lisäksi vanhojen sähköverkon vikaviestipalveluiden ja Keskuskauppakamarin dokumenttikäsittelyn ylläpitotöitä\\
    \faIcon{carrot} \texttt{Bash}, \texttt{Apache}, \texttt{ANT}, \texttt{PSQL}, \texttt{Glassfish}, \texttt{Wildfly}, \texttt{Liferay}, \texttt{Alfresco}}
  \twentyitem{2017-2018}{Junior Developer}{Innotect Oy}
    {Microsoft Azure pohjaisten sovellusten kehitystä ja elektroniikkalaitteiden suunnittelua ja ohjelmointia.\\
    \faIcon{carrot} \texttt{Azure Function Apps (C\#)}, \texttt{BlinkStick (Python, C\#)}, \texttt{Legato (C)}, \texttt{IoT}}
  \twentyitem{2015-2016}{Harjoittelija}{ABB Oy}
    {Matalajännitetaajuusmuuttajatuotteiden hallintatiimi.\\
		Kilpailija-analyysia, tietokantahallintaa, jne.}
  \twentyitem{2014}{Harjoittelija}{ABB Oy}
    {Taajuusmuuttajatuotannon linjastotyötä (ACS800).}
  \twentyitem{2013}{HMI Suunnittelija}{Elmont Oy}
    {Käyttöliittymäsuunnittelua tehdaslinjaston tilaushallintasovellukselle.\\
    \faIcon{carrot} \texttt{C\#}, \texttt{SQL}
    }
	%\twentyitem{<dates>}{<title>}{<location>}{<description>}
\end{twenty}



%----------------------------------------------------------------------------------------
%	 EDUCATION
%----------------------------------------------------------------------------------------

\section{Koulutus}

\begin{twenty} % Environment for a list with descriptions
  \twentyitem{2015-2022}{Diplomi-insinööri Automation and Control Engineering}{Aalto-yliopisto}
  {Pääaine: Control, robotics and autonomous systems \\
  maustettuna Computer, Communication and Information Science ohjelman kursseilla \\
  Sivuaine: Sovelluskehitys \\
  Diplomityö: \emph{Machine learning approach to support ticket forecasting from software logs}}
  \twentyitem{2011-2015}{Tekniikan kandidaatti Automaatio- ja systeemitekniikka}{Aalto-yliopisto}
  {Sivuaine: Sovelluskehitys \\
	Kandidaatin työ: \emph{Offline-programming of industrial robots}}
  \twentyitem{2009}{Lukio}{Lahden Yhteiskoulu}{Ylioppilastutkinto}
  %\twentyitem{<dates>}{<title>}{<location>}{<description>}
\end{twenty}

%----------------------------------------------------------------------------------------
%	 OTHER INFORMATION
%----------------------------------------------------------------------------------------



\begin{minipage}[t]{0.5\textwidth}
  \vspace{-\baselineskip} % Required for vertically aligning minipages

  \section{OS}

  \textbf{Windows} - Natiivi \\
  \textbf{Ubuntu} - Erinomainen \\
  \textbf{Ubuntu CLI} - Asiantuntija


\end{minipage}
\hfill
\begin{minipage}[t]{0.5\textwidth}
  \vspace{-\baselineskip} % Required for vertically aligning minipages

  \section{Kielitaito}

  \textbf{Suomi} - Äidinkieli \\
  \textbf{Englanti} - Erinomainen \\
  \textbf{Ruotsi} - Tyydyttävä \\

\end{minipage}


%----------------------------------------------------------------------------------------
%	 MISC
%----------------------------------------------------------------------------------------

\section{Lisäksi}


\subsection{Ammatillisesti sanoen}

Olen kiinnostunut Full-stack kehityksestä, pilviteknologioista, koneoppimisesta ja asiakasrajapinnassa työskentelystä.
Olen aina pitänyt ihmisten kanssa työskentelystä ja todellisten ongelmien ratkomisesta omaksumani kokemuksen pohjalta.
Olipa kyseessä sitten kehitysprojektit, yrityksen sisäiset prosessit,
tai asiakkaiden tilaustyöt,
olen parhaimmillani päästessäni selvittämään todellista tarvetta ongelman taustalla
ja kehittäessäni siihen parhaan mahdollisen ratkaisun.

\subsection{Harrastelemisia}
	Pidän itseni haastamisesta erilaisissa urheilulajeissa.
	Useimmiten harjoittelen parkouria ja telinevoimistelua,
	mutta nautin myös lumilautailusta ja vesiurheilusta.
	Urheilun lisäksi pelailen videopelejä, säädän elektroniikan ja mikrokontrollerien parissa,
	sekä rakentelen käyttö- ja koriste-esineitä jämäosista.


%----------------------------------------------------------------------------------------
%	 SECOND PAGE EXAMPLE
%----------------------------------------------------------------------------------------

%\newpage % Start a new page

%\makeprofile % Print the sidebar

%\section{Other information}

%\subsection{Review}

%Alice approaches Wonderland as an anthropologist, but maintains a strong sense of noblesse oblige that comes with her class status. She has confidence in her social position, education, and the Victorian virtue of good manners. Alice has a feeling of entitlement, particularly when comparing herself to Mabel, whom she declares has a ``poky little house," and no toys. Additionally, she flaunts her limited information base with anyone who will listen and becomes increasingly obsessed with the importance of good manners as she deals with the rude creatures of Wonderland. Alice maintains a superior attitude and behaves with solicitous indulgence toward those she believes are less privileged.

%\section{Other information}

%\subsection{Review}

%Alice approaches Wonderland as an anthropologist, but maintains a strong sense of noblesse oblige that comes with her class status. She has confidence in her social position, education, and the Victorian virtue of good manners. Alice has a feeling of entitlement, particularly when comparing herself to Mabel, whom she declares has a ``poky little house," and no toys. Additionally, she flaunts her limited information base with anyone who will listen and becomes increasingly obsessed with the importance of good manners as she deals with the rude creatures of Wonderland. Alice maintains a superior attitude and behaves with solicitous indulgence toward those she believes are less privileged.

%----------------------------------------------------------------------------------------

\end{document}
